\documentclass[10pt,]{article}
\usepackage[sc, osf]{mathpazo}
\usepackage{amssymb,amsmath}
\usepackage{ifxetex,ifluatex}
\usepackage{fixltx2e} % provides \textsubscript
\ifnum 0\ifxetex 1\fi\ifluatex 1\fi=0 % if pdftex
  \usepackage[T1]{fontenc}
  \usepackage[utf8]{inputenc}
\else % if luatex or xelatex
  \ifxetex
    \usepackage{mathspec}
  \else
    \usepackage{fontspec}
  \fi
  \defaultfontfeatures{Ligatures=TeX,Scale=MatchLowercase}
\fi
% use upquote if available, for straight quotes in verbatim environments
\IfFileExists{upquote.sty}{\usepackage{upquote}}{}
% use microtype if available
\IfFileExists{microtype.sty}{%
\usepackage{microtype}
\UseMicrotypeSet[protrusion]{basicmath} % disable protrusion for tt fonts
}{}
\usepackage[margin=1in]{geometry}




\setlength{\emergencystretch}{3em}  % prevent overfull lines
\providecommand{\tightlist}{%
  \setlength{\itemsep}{0pt}\setlength{\parskip}{0pt}}
\setcounter{secnumdepth}{0}
% Redefines (sub)paragraphs to behave more like sections
\ifx\paragraph\undefined\else
\let\oldparagraph\paragraph
\renewcommand{\paragraph}[1]{\oldparagraph{#1}\mbox{}}
\fi
\ifx\subparagraph\undefined\else
\let\oldsubparagraph\subparagraph
\renewcommand{\subparagraph}[1]{\oldsubparagraph{#1}\mbox{}}
\fi

% Now begins the stuff that I added.
% ----------------------------------

% Custom section fonts
\usepackage{sectsty}
\sectionfont{\rmfamily\mdseries\large\bf}
\subsectionfont{\rmfamily\mdseries\normalsize\itshape}


% Make lists without bullets
%\renewenvironment{itemize}{
%  \begin{list}{}{
%    \setlength{\leftmargin}{1.5em}
%  }
%}{
%  \end{list}
%}


% Make parskips rather than indent with lists.
\usepackage{parskip}
\usepackage{titlesec}
\titlespacing\section{0pt}{12pt plus 4pt minus 2pt}{4pt plus 2pt minus 2pt}
\titlespacing\subsection{0pt}{12pt plus 4pt minus 2pt}{4pt plus 2pt minus 2pt}

% Use fontawesome. Note: you'll need TeXLive 2015. Update.
\usepackage{fontawesome}

% Fancyhdr, as I tend to do with these personal documents.
\usepackage{fancyhdr,lastpage}
\pagestyle{fancy}
\renewcommand{\headrulewidth}{0.0pt}
\renewcommand{\footrulewidth}{0.0pt}
\lhead{}
\chead{}
\rhead{}
\lfoot{
\cfoot{\scriptsize  Grace Brewster Murray Hopper - CV -  }}
\rfoot{\scriptsize \thepage/{\hypersetup{linkcolor=black}\pageref{LastPage}}}

% Always load hyperref last.
\usepackage{hyperref}
\PassOptionsToPackage{usenames,dvipsnames}{color} % color is loaded by hyperref

\hypersetup{unicode=true,
            pdftitle={Grace Brewster Murray Hopper:  CV (Curriculum Vitae)},
            pdfauthor={Grace Brewster Murray Hopper},
            pdfkeywords={RMarkdown, academic CV, template},
            colorlinks=true,
            linkcolor=blue,
            citecolor=Blue,
            urlcolor=blue,
            breaklinks=true, bookmarks=true}
\urlstyle{same}  % don't use monospace font for urls

\begin{document}


\centerline{\huge \bf Grace Brewster Murray Hopper}

\vspace{2 mm}

\hrule

\vspace{2 mm}

\moveleft.5\hoffset\centerline{Computer Scientist \& United States Navy Rear Admiral}
\moveleft.5\hoffset\centerline{Address · Arlington · VA 22205}
\moveleft.5\hoffset\centerline{ \faEnvelopeO \hspace{1 mm} \href{mailto:}{\tt \href{mailto:FLOW-MATIC@cobol}{\nolinkurl{FLOW-MATIC@cobol}}} \hspace{1 mm}  \faGithub \hspace{1 mm} \href{http://github.com/hopper}{\tt hopper} \hspace{1 mm}    \faGlobe \hspace{1 mm} \href{http://wikipedia.org/wiki/Grace\_Hopper}{\tt wikipedia.org/wiki/Grace\_Hopper}    | \emph{Updated:} \today}

\vspace{2 mm}

\hrule


\section{EDUCATION}\label{education}

\emph{Yale University}, Ph.D.~Mathematics \hfill 1934

\emph{Yale University}, M.S. Mathematics \hfill 1930

\emph{Vassar College}, B.S. Mathematics \& Physics \hfill 1928

\section{EMPLOYMENT}\label{employment}

\emph{United States Naval Reserve:}

\begin{quote}
Naval Reserve Midshipmen's School \hfill 1943--1944 \emph{Bureau of
Ships Computation Project (Harvard):}
\end{quote}

\begin{quote}
Research Fellow \hfill 1945-1949 Lieutenant, Jr Grade \hfill 1944
\end{quote}

\emph{Eckert--Mauchly Computer Corporation:} \textgreater{} Senior
Mathematician \hfill 1949---1967 \emph{Navy Programming Languages Group,
Navy Office of Information Systems Planning:}

\begin{quote}
Director (and promoted to Captain) \hfill 1967--1977\\
\# TEACHING
\end{quote}

\emph{Associate Professor, Mathematics, Vassar} \hfill 1931--1941

\begin{quote}
Teach students about math, began career in programming and largely
developed design and implementation of a computer compiler.
\end{quote}

\emph{Director, Navy Programming Languages Group} \hfill 1967--1977

\begin{quote}
She developed validation software for COBOL and its compiler as part of
a COBOL standardization program for the entire Navy. Hopper advocated
for the Defense Department to replace large, centralized systems with
networks of small, distributed computers. Any user on any computer node
could access common databases located on the network. She developed the
implementation of standards for testing computer systems and components,
most significantly for early programming languages such as FORTRAN and
COBOL. The Navy tests for conformance to these standards led to
significant convergence among the programming language dialects of the
major computer vendors. In the 1980s, these tests (and their official
administration) were assumed by the National Bureau of Standards (NBS),
known today as the National Institute of Standards and Technology
(NIST). (Source: \url{https://en.wikipedia.org/wiki/Grace_Hopper}) \#
PUBLICATIONS
\end{quote}

\subsection{\texorpdfstring{\textbf{Journal
Articles}}{Journal Articles}}\label{journal-articles}

G. M. Hopper and O. Ore. 1934. ``\emph{New types of irreducibility
criteria}.'' Bull. Amer. Math. Soc. 40 (216).

\subsection{\texorpdfstring{\textbf{Books}}{Books}}\label{books}

Beyer, Kurt W. 2009. \emph{Grace Hopper and the Invention of the
Information Age}. Cambridge, Massachusetts: MIT Press. ISBN
978-0-262-01310-9.

Williams, Kathleen Broome. 2004. \emph{Grace Hopper: Admiral of the
Cyber Sea} (1st ed.). Annapolis, Maryland: Naval Institute Press. ISBN
978-1-55750-952-9.

\section{PUBLIC MEDIA}\label{public-media}

``Nano-seconds'' lecture by Grace Hopper.
\url{https://www.youtube.com/watch?v=JEpsKnWZrJ8}

\section{AWARDS}\label{awards}

\textbf{1964} \emph{Society of Women Engineers Achievement Award}, the
Society's highest honor, ``In recognition of her significant
contributions to the burgeoning computer industry as an engineering
manager and originator of automatic programming systems.''

\textbf{1969} \emph{Data Processing Management Association Man of the
Year} (now called the Distinguished Information Sciences Award)

\textbf{1973} \emph{Distinguished Fellow of the British Computer
Society}. First American and the first woman of any nationality to be
given award.

\textbf{1982} \emph{American Association of University Women Achievement
Award}

\textbf{1987} \emph{Computer History Museum Fellow Award Recipient}
First to receive this award, for contributions to the development of
programming languages, for standardization efforts, and for lifelong
naval service.

\textbf{1991}: \emph{National Medal of Technology}

\textbf{1996}: \emph{USS Hopper (DDG-70) was launched}. Nicknamed
Amazing Grace, it is on a very short list of U.S. military vessels named
after women.

\textbf{2009}: The Department of Energy's National Energy Research
Scientific Computing Center named its flagship system ``Hopper''

\textbf{2016}: Posthumously awarded a \emph{Presidential Medal of
Freedom} for her accomplishments in the field of computer science

\end{document}
